\section{Introduction}\label{sec:intro}

Sonar Workbench is a suite of Matlab tools for the design and analysis of sonar systems. Version 1.0 focused on array construction and beam pattern construction and analysis. Version 2.0 added TEAMS interface support. Version 3.0 vectorized position and orientation parameters for increased efficiency. Sonar Workbench includes multiple element types, from which the user constructs an array with arbitrary element position and orientation. The user defines complex weights for each element to create a beam with the desired shape, which can be evaluated at any elevation and azimuthal angles, plotted in 3D space, and measured for beam width and directivity index.

This user's guide is intended to introduce a new user to Sonar Workbench, the array theory it implements, and its usage in the Matlab environment. Much of the content is adapted from \emph{An Introduction to Sonar Systems Engineering} \cite{Ziomek}, which is recommended as a companion resource for the user interested in understanding more of the theory. Section~\ref{sec:coord} introduces the coordinate system and reference frames used in Sonar Wrokbench. Section~\ref{sec:element} explains how to define elements and lists the built-in element types. Section~\ref{sec:array} describes the process to build arrays consisting of uniform or mixed element types. Section~\ref{sec:beam} defines beams and demonstrates a method for calculating amplitude and phase weights for conventional beamforming. Section~\ref{sec:analysis} demonstrates how Sonar Workbench implements these concepts and conventions and shows the user how to use it for their own analysis.

The example element, array, and beam definitions used in this guide can be found in the \texttt{test} folder, along with a script \texttt{CreateSampleBeam.m} the user can execute to demonstrate the features of Sonar Workbench. 