\chapter{Elements}\label{ch:element}

The transducer element is the fundamental building block for arrays and the starting point for analysis in Sonar Workbench. For the purpose of generating and analyzing beam patterns, specific electromechanical transduction methods do not need to be modeled; instead, the element's acoustic properties can be captured by modeling vibrations of the element's wetted surface. Hereafter, references to an element's geometry refer to the geometry of the element's wetted surface or face. Sonar Workbench treats each element as a uniformly vibrating surface, which is to say that it only models each surface's fundamental mode of vibration. Analyzing element response requires first, defining the element geometry, and second, evaluating that geometry at a specific acoustic wavelength to produce an element pattern.

\section{Element definition}

Sonar Workbench includes support for the element types listed in Table~\ref{tab:ElementTypes}. 

\begin{table}[!ht]
	\begin{center}
		\caption{Included element types}
		\label{tab:ElementTypes}
		\begin{tabular}{c|c} 
			\textbf{Element Type} & \textbf{Enumeration} \\
			\hline
			Omnidirectional  & 0 \\
			Cosine & 1 \\
			Uniform Line & 2 \\
			Circular Piston & 3 \\
			Rectangular Piston & 4 \\
			Hexagonal Piston & 5 \\
			Annular Piston & 6 \\
		\end{tabular}
	\end{center}
\end{table}

An element structure holds the parameters that define the element geometry. The element structure must contain the \texttt{.type} field with an integer corresponding to the elements listed in Table~\ref{tab:ElementTypes}, which selects a \texttt{.m} file that generates the corresponding element pattern. Table~\ref{tab:ElementTypes} lists the built-in element type enumerations.

The field \texttt{.baffle} dictates whether the element should be baffled by the element frame $y$-$z$ plane (e.g. arrays mounted to platforms) or unbaffled (e.g. towed arrays, sonobuoys). The omnidirectional, uniform line, and cosine elements can all be used with and without baffling. The piston elements' element patterns are all derived from equations assuming the piston is mounted in an infinite rigid baffle; therefore, it is recommended that the user set these element's \texttt{.baffle=1} for best results. Diffraction effects caused by finite baffle dimensions are beyond the scope of Sonar Workbench. 

The field \texttt{.params\_m} is a three-element column vector defining the element geometry. A different number of parameters are required to define the shape of different element types. The components of \texttt{.params\_m} for each element type are listed in Table~\ref{tab:ElementParams}. As indicated by the field name, dimension units are in meters.

\begin{table}[!ht]
	\begin{center}
		\caption{Element parameter fields}
		\label{tab:ElementParams}
		\begin{tabular}{c|c|l} 
			\textbf{Element Type} & \textbf{Vector Component} & \textbf{Description} \\
			\hline
			Omnidirectional & \texttt{1} & radius (m)\\
			\hline
			Cosine & \texttt{1} & circumscribed sphere radius (m)\\
			\hline
			\multirow{3}{*}{Uniform Line} & \texttt{1} & length along x-axis (m)\\
			& \texttt{2} & length along y-axis (m)\\
			& \texttt{3} & length along z-axis (m)\\
			\hline
			Circular Piston & \texttt{1} & radius (m)\\
			\hline
			\multirow{2}{*}{Rectangular Piston} & \texttt{1} & width (m)\\
			& \texttt{2} & height (m)\\
			\hline
			Hexagonal Piston & \texttt{1} & inscribed circle radius (m)\\	
			\hline
			\multirow{2}{*}{Annular Piston} & \texttt{1} & outer radius (m)\\
			& \texttt{2} & inner radius (m)\\
		\end{tabular}
	\end{center}
\end{table}

For a line element, only one of the \texttt{params\_m} components should be nonzero, corresponding to the axis along which the line array is aligned. Listing~\ref{lst:SampleElement} shows the contents of \texttt{SampleElement.m}, which defines a rectangular piston element.

\lstinputlisting[caption={\texttt{SampleElement.m}},label={lst:SampleElement}]{../../test/SampleElement.m}

\section{Element patterns}

Element geometry, coupled with an acoustic wavelength, defines the element pattern. The element pattern is the element's far-field directional response as a function of wavelength, azimuth and elevation, and it can be thought of as a spatial filter. Elements are assumed to be transducers, capable of transmitting and receiving sound, so there is no distinction between transmit and receive element patterns. 

Sonar Workbench uses the acoustic wavelength, $\lambda$, to calculate element patterns because it combines the frequency and sound speed into a single term. It is related to sound speed $c$, frequency $f$ in Hz or $\omega$ in rad/s, and wavenumber $k$ in m$^{-1}$ by
\begin{equation*}
\lambda = \frac{c}{f} = \frac{2\pi{c}}{\omega} = \frac{2\pi}{k}.
\end{equation*}

At its most general, the element pattern is the three-dimensional Fourier transform of the element's complex aperture function, $A(\lambda,x,y,z)$,
\begin{equation}
E(\lambda,\theta,\psi) = \int_{-\infty}^\infty\int_{-\infty}^\infty\int_{-\infty}^\infty A(\lambda,x,y,z)e^{j2\pi\left(\frac{\cos\theta\cos\psi}{\lambda}x + \frac{\cos\theta\sin\psi}{\lambda}y + \frac{\sin\theta}{\lambda}z\right)}dxdydz,\label{eq:VolumetricFT}
\end{equation}
For the piston elements, the element pattern reduces to a two-dimensional Fourier transform, since the element face lies in the $y$-$z$ plane,
\begin{equation}
E_{piston}(\lambda,\theta,\psi) = \int_{-\infty}^\infty\int_{-\infty}^\infty A(\lambda,y,z)e^{j2\pi\left(\frac{\cos\theta\sin\psi}{\lambda}y + \frac{\sin\theta}{\lambda}z\right)}dydz,\label{eq:PlanarFT}
\end{equation}
and for the uniform line array, it further reduces to a one-dimensional Fourier transform,
\begin{equation}
E_{line}(\lambda,\theta,\psi) = \int_{-\infty}^\infty A(\lambda,y)e^{j2\pi\frac{\cos\theta\sin\psi}{\lambda}y}dy,\label{eq:LinearFT}
\end{equation}
for a line array aligned with the $y$ axis. The finite element extents make the integration limits finite. For the simple elements included with Sonar Workbench, the assumption of uniform surface motion means that the aperture function is real-valued and equal to 1 over the entire element surface. This simplifies the integration for certain element geometries. These integrals have analytic solutions, which Sonar Workbench uses instead of evaluating the integrals numerically.

Element patterns are normalized such that they have unity gain along their maximum response axis. \figurename~\ref{fig:ElementPatterns} shows element patterns for each of the included element types listed in Table~\ref{tab:ElementTypes} for a wavelength equal to half of the element's maximum dimension. The uniform line element is aligned with the $y$ axis, and the cosine element is aligned with the $x$ axis. Note that the omnidirectional and cosine element patterns do not depend on wavelength.

\begin{sidewaysfigure}[!ht]
\begin{center}
\includegraphics[width=7.75in]{ElementPatterns}
\caption{\label{fig:ElementPatterns}Element patterns for included element types (magnitude in dB)}
\end{center}
\end{sidewaysfigure}
